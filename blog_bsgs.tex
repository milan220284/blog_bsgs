
\documentclass[fleqn,10pt]{olplainarticle}
% Use option lineno for line numbers 

\title{Baby step - Giant step algorithm}

\author[1]{Milan Pavlović, LogN team}

\keywords{discret log, algorithm, cryptography, Baby step - Giant step, LogN}

\begin{abstract}
This is another blog about topics that we at LogN team meet 
while doing research in area of Zero Knowladge. 
Point of interest in this blog is Baby step - Giant step algorithm. 

\end{abstract}

\begin{document}

\flushbottom
\maketitle
\thispagestyle{empty}

\section*{Table of contents}

\begin{itemize}[noitemsep] 
    \item Intro: Elliptic curves and DLP
    \item Algorithm
    \item Example
    \item Conclusion
    \item Code
    \item Resources
\end{itemize}


\section*{Intro: Elliptic curves and DLP}

\par \ \ \ \ \ \ Elliptic curve $E(\mathbb{F})$ over field $\mathbb{F}$ have structure
 of additive group, more specific abelian group. 
 Details of that structure demands whole another blog, 
 interested reader is refered to references on the end of the blog. 

Problem that lies in basis of elliptic curve cryptography is: 
\vspace{0.2cm}

Given two points, say $P$ and $Q$ in subgroup $G$ of elliptic 
curve group $E(\mathbb{F}_q)$ over finite field $\mathbb{F}_q$, 
find integer $k$ such that $P=[k]Q$ 
(notation widely accepted for additon on elliptic curve).
\vspace{0.2cm}

Problem cited above is called elliptic curve discrete logarithm 
problem (ECDLP) and there is no known polynomial time algorithm 
for solving that can run on a classical computer. 
It is "elliptic" because we are doing calculations on elliptic 
curve, "discret" because coordinates of points are integers 
from finite field and "logarithm" because it is analogous to classical 
mathematical logarithm.

Discrete logarithm problem(DLP) is also known in other 
cryptosystems such as the Digital Signature Algorithm (DSA), 
the Diffie-Hellman key exchange (DH) and the ElGamal algorithm. 
There is reason why they share the name, they represent similar 
problems. Difference between ECDLP and DLP is that in ECPLD we 
work with additive group of ellpitic curve while in DLP one 
works with multiplicative group. The DLP problem can be stated 
as:
\vspace{0.2cm}

Given $y$ and $g$ from group and integer $p$, find $x$ such that 
$y=g^x \ (\textrm {mod}\ p)$.
\vspace{0.2cm}

In this blog we'll present how one can solve DLP for prime p, 
similar ideas then can be applied for ECDLP. We choose this path 
for clarity of presentation, to avoid dealing with elliptic 
curves.

What makes elliptic curves useful in cryptography is that, 
as of today, the discrete logarithm problem for elliptic curves 
seems to be "harder" if compared to other similar problems used 
in cryptography (intuitively, the reason is complexity of 
elliptic curve).  This implies that we need fewer bits to 
achieve the same level of security as with other cryptosystems.

\section*{Algorithm}

\par \ \ \ \ \ \ We'll now describe algorithm used to solve DLP, which is, due to 
Daniel Shanks, called Baby step - Giant step. This algorithm can 
be applied to any finite cyclic abelian group. Depending of use 
case some modifications are posible.
\vspace{0.2cm}

Asume we have public cyclic group $G=\langle g \rangle$ of prime order $p$. 
Given $y\in G$, we are asked to find value $x\in\mathbb{Z}_p$ 
such that $y=g^x\ (\textrm {mod}\ p)$.
\vspace{0.2cm}

Idea behind algorithm is often encountered idea of divide and 
conquare. We first calculate $k=[\sqrt{p}]+1$. Then we write:
$$x=x_0+x_1\cdot k,$$
where $0\leq x_0,\ x_1\leq p$.
To continue, we calculate baby steps:
$$g_i = g^i,\ 0\leq i < k.$$
The values $g_i$ are then stored in array $X$. To compute and store these values requires $\mathcal{O}(\sqrt{p})$ time and space. We then proceed to Giant steps:
$$y_j = y\cdot g^{-jk},\ 0\leq j< k.$$
After calculating each $y_j$, we try to find that value in array $X$. If such match is found, say $X[i]=y_j$, we have solution to our DLP and that solution is:
$$x=i+j\cdot k\ (\textrm {mod}\ p).$$
We can analyze last assertion a bit. Match  $X[i]=y_j$ is equivalent to:
$$g^i=y\cdot g^{-jk}$$
$$g^{i+j\cdot k}=y \ (\textrm{mod}\ p).$$
From last equation we read solution to our discrete logarithm problem, $x=i+j\cdot k\ (\textrm {mod}\ p).$
Notice that Giant step requires at most $\mathcal{O}(\sqrt{p})$ time. Hence, the overall time and space complexity of the Baby step - Giant step algorithm is $\mathcal{O}(\sqrt{p})$. This means that if we want our problem to bi $2^{128}$ difficult, we need to take prime of order $2^{256}$.

\section*{Example}

\par \ \ \ \ \ \ As an ilustration let's take finite field $\mathbb{F}_{509}$ 
and take a closer look on it's multiplicative subgroup $G$ of 
order 127 generated by element $g=17$. Further on, integer 
$y=438$ is given and we want to find $x$ such that:
$$438=17^x\ (\textrm {mod}\ 509),$$
i.e. our task is to solve DLP.

Following presented algorithm, we first calculate ceiling $k=[\sqrt{509}]+1=23$. Then, we proceed to Baby steps:
$$g_i = 17^i,\ 0\leq i < k.$$
Values $g_i$ are stored in array $X$ which is given in table below in format $i: 17^i\ (\textrm{mod}\ 509).$
% \begin{figure}[ht]
% \centering
% \includegraphics[width=0.7\linewidth]{frog}
% \caption{An example image of a frog.}
% \label{fig:view}
% \end{figure}

\begin{table}[ht]
\centering
\begin{tabular}{|l|l|l|l|}
  0: 1   & 6: 280  & 12: 14 & 18: 357\\
  1: 17  & 7: 179 & 13: 238 & 19: 470 \\
  2: 289 & 8:  498 & 14: 483 & 20: 355\\
  3: 332 & 9:  322 & 15: 67 & 21: 436 \\
  4: 45 & 10: 384 & 16: 121 & 22: 286\\
  5: 256 & 11: 420 & 17: 21 
\end{tabular}
\caption{\label{tab:widgets}Array X in format $i: 17^i\ (\textrm{mod}\ 509).$}
\end{table}


Next step is Giant step. We calculate
$$y_j = 438\cdot g^{-k\cdot j},\ 0\leq j< k,$$
and after each calculation try to find same value in table above. In this example, calculations gives us:

$y_0 = 438$, which is not value from table; 

$y_1 = 199$, which also is not found in table; 

$y_2=238$ is in table. We memorize index of $y$, that is $j=2$.

Value $238$ is calculated in Baby step for $i=13$, so solution 
to our DLP is $x=13+2\cdot 23=59.$ You can check that 
$17^{59}=438\ (\textrm{mod}\ 509)$ indeed.

Optimization can be made if one knows that $g$ is in some 
subgroup of $\mathbb{F}_p$. As we know here that $g$ is in 
subgroup $G$ of order 127, order of that subgroup can be used 
for calculating ceiling $k=[\sqrt{127}]+1=12$ instead of order 
509 of whole group which yields $k=[\sqrt{509}]+1=23$. 
This optimization has its own cost, we have to determine order 
of $g$.

\section*{Conclusion}

\par \ \ \ \ \ \ There are a lot applications of Baby step - Giant step algorithm, 
and some modifications. One limitation of algorithm is that it 
has $\mathcal{O}(\sqrt{p})$ space complexity, beside 
$\mathcal{O}(\sqrt{p})$ time complexity. It's a big drawback and 
reason why DLP is hard for todays computers. One optimization 
in that direction is Pollard's Rho algorithm. Other 
generalization is Pohlig-Hellman algorithm which is used when 
integer $p$ in DLP is not prime. This algorithm reduces the DLP 
from a group of composite order to subgroups of prime order, 
then solve DLP in each subgroup. Chinese Remainder Theorem is 
used to reconstruct solution of original problem. One of 
important conclusions from Pohlig-Hellman algorithm is that DLP 
in group $G$ is as hard as it is DLP in its largest subgroup of 
prime order (which dominates in time and space complexity). 
This idea lies in known Pohlig Hellman small subgroup attacks.
\section*{Code}

\par \ \ \ \ \ \ Small Python script for our example (calculations in our 
example are tested in SageMath too):
\begin{lstlisting}[language=Python]
    from math import sqrt

    def babyGiant(g, y, p, option = False):
        '''
        Returns x s.t. y = g^x mod p for a prime p.
        optional argument for printing intermediate steps
        '''
        # calculating ceiling
        k = round(sqrt(p - 1))
    
        X = [None] * k
    
        # Baby steps
        for i in range(k):
            X[i] = pow(g, i, p)
        
        #Calculating inverse using Fermat's little theorem
        inv = pow(g, k * (p - 2), p)
        
        if option:
            print("X: ")
            for i in range(k): print(i, ':', X[i])
            Y = [None] * k
            
        # Giant step, search for match in the list
        for j in range(k):
            tmp = (y * pow(inv, j, p)) % p
            if option: Y[j] = tmp
    
            if tmp in X:
                if option: print("Y: ", Y[:j+1])
                return X.index(tmp) + j * k 
        # Solution not found
        return None
    
    print(babyGiant(17, 438, 509, True))
\end{lstlisting}

\section*{Resources}

\begin{itemize}[noitemsep] 
    \item Cryptography: An Introduction, Nigel Smart 
    \item  Rational Points on Elliptic Curves, Joseph H. Silverman and 
    John T. Tate
    \item Pairings for beginners, Craig Costello
    \item https://andrea.corbellini.name
    \item https://www.sagemath.org
\end{itemize}


\end{document}